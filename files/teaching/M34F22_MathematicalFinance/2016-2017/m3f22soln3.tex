\documentclass[12pt]{article}
\usepackage{amsfonts}
\usepackage{amsmath}
\usepackage{pifont}
\usepackage{natbib}
\usepackage{color}
\def\bibfont{\footnotesize}
\setlength{\bibsep}{0pt plus 0.3ex}

\usepackage{url}

\begin{document}
\def\R{\mathbb{R}}
\def\C{\mathbb{C}}
\def\Z{\mathbb{Z}}
\def\N{\mathbb{N}}
\def\Q{\mathbb{Q}}
\def\D{\mathbb{D}}
\def\Sp{{\mathbb{S}}}
\def\T{\mathbb{T}}
\def\H{\mathbb{H}}
\def\hb{\hfil \break}
\def\ni{\noindent}
\def\i{\indent}
\def\a{\alpha}
\def\b{\beta}
\def\e{\epsilon}
\def\d{\delta}
\def\D{\Delta}
\def\G{\Gamma}
\def\g{\gamma}
\def\l{\lambda}
\def\m{\mu}
\def\s{\sigma}
\def\Si{\Sigma}
\def\th{\theta}
\def\z{\zeta}
\def\p{\partial}
\def\o{\omega}
\def\O{\Omega}
\def\t{\tau}
\def\L{\it \char'44}
\def\F{\mathcal{F}}
\def\B{\mathcal{B}}
\def\C{\mathcal{C}}
\def\half{\frac{1}{2}}
\ni m3f33soln3 \\
\begin{center}
{\bf M3F22 SOLUTIONS 3.  3.11.2017} 
\end{center}

\ni Q1. (i) In spherical polar coordinates $(r, \t, \phi)$ ($r$: distance from centre, range $0$ to $\infty$; $\t$: colatitude (= $\half \pi$ - latitude), range 0 to $\pi$; $\phi$ longitude, range 0 to $2 \pi$): increase $r$ to $r + dr$, etc.  The element of volume $dV$ is a (to first order) cuboid, of sides $dr$ ("up"), $r d \t$ ("South"), $r \sin \t d \phi$ ("East") (draw a diagram -- or consult a textbook if you need one!)  So
$$
dV = dr.r d \t.r \sin \t d \phi = r^2 \sin \t dr d\t d \phi.
$$
So
$$
V = \int_0^r r^2 dr \int_0^{2 \pi} d \phi \int_0^{\pi} \sin \t d \t
= \frac{1}{3} r^3. 2 \pi [- \cos \t]_0^{\pi} = \frac{2 \pi}{3} r^3 [-(-1) - (-1)] = 4 \pi r^3/3.
$$
(ii) Holding $r$ fixed,
$$
dS = r^2 \sin \t.d\t d \phi.
$$
So
$$
A = r^2 \int_0^{2 \pi} d \phi \int_0^{\pi} \sin \t d \t = r^2.2 \pi. 2 = 4 \pi r^2,
$$
by above. \\
(iii)  To first order,
$$
dV = S dr: \qq S = dV/dr, \qq V = \int_0^r S dr
$$
(`flattening out' the spherical shell: volume = area $\times$ thickness: the curvature effects are second-order).  So (i), (ii) are equivalent: ((ii) follows from (i) by differentiating, and (i) from (ii) by integrating. \\

\ni Q2.  This follows by the same method as the area of an ellipse $A = \pi ab$: wlog $a \geq b \geq c$.  Compress [squash] the $x$- and $y$-axes in the ratios $a/c$, $b/c$, to get a sphere of radius $c$.  This has volume $4 \pi c^3/3$.  Now dilate [unsquash] the $x$- and $y$-axes in the ratios $a/c$, $b/c$, to get volume
$$
V = \frac{4 \pi c^3}{3}.\frac{a}{c}.\frac{b}{c} = \frac{4 \pi abc}{3}.
$$

Q3. (i) Choose the vertex $V$ as origin, and the $z$-axis vertical -- the perpendicular from $V$ to the horizontal base (with $z$ going downwards, if we draw the tetrahedron the usual way).  Slice the volume into thin horizontal slices.  The area of the slice between $z$ and $z + dz$ is $A (z/h)^2$, by similarity.  So
$$
V = \int_0^h A (z/h)^2 dz = A h^{-2} \int_0^h z^2 dz = A h^{-2}.h^3/3:
$$
$$
V = Ah/3.
$$
(ii) Similarly in the general case: the above does not use that the base is triangular. \\

\ni Q4. (i) The range between $x$ and $x + dx$ generates volume $dV = \pi y^2 dx = \pi f(x)^2 dx$.  Integrate this from $a$ to $b$. \\
(ii) The semicircle on base $[-r,r]$ is $y = f(x) = \sqrt{r^2 - x^2}$.  This generates te sphere on revolution, giving
$$
V = \int_{-r}^r \pi (r^2 - x^2) dx = \pi [r^2 x - \frac{1}{3} x^3]_{-r}^r = \pi r^3 [1 - \frac{1}{3} - (-1) + (- \frac{1}{3})] = \pi r^3 (2 - \frac{2}{3}) = 4 \pi r^3/3.
$$

\ni Q5  (Georges BOULIGAND, 1935).  {\it First Proof}.  For the region $S_1$ with area $A_1$ with base the hypotenuse, side 1: use cartesian coordinates to approximate its area, arbitrarily closely, by decomposing it into small squares of area
$dA_1 = dx dy$. \\
\i For each such small square on side 1, construct similar small squares on sides 2 and 3, of areas $dA_2$, $dA_3$. \\
\i By Pythagoras' theorem, $dA_1 = dA_2 + dA_3$. \\
\i Summing, we get $A_1 = A_2 + A_3$ arbitrarily closely, and so exactly. \\
{\it Second Proof}.  Drop a perpendicular from the right-angled vertex to the hypotenuse.  This splits the `big figure' into two `smaller figures', each similar to it.  With $l_1$ the length of the hypotenuse and $l_2$, $l_3$ those of the other two sides, by similarity lengths scale by $l_2/l_1$, $l_3/l_1$ on going from the big figure to the smaller ones, so areas scale by $(l_2/l_1)^2$, $(l_3/l_1)^2$.  So $A_2 + A_3 = A_1[(l_2/l_1)^2 + (l_3/l_1)^2] = A_1(l_2^3 + l_3^2)/l_1^2, = A_1$ by Pythagoras' theorem. // \\

\hfil NHB \break


\end{document}