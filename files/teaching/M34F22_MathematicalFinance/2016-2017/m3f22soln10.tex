\documentclass[12pt]{article}
\usepackage{amsfonts}
\usepackage{amsmath}
\usepackage{pifont}
\usepackage{natbib}
\usepackage{color}
\def\bibfont{\footnotesize}
\setlength{\bibsep}{0pt plus 0.3ex}

\usepackage{url}

\begin{document}
\def\R{\mathbb{R}}
\def\C{\mathbb{C}}
\def\Z{\mathbb{Z}}
\def\N{\mathbb{N}}
\def\Q{\mathbb{Q}}
\def\D{\mathbb{D}}
\def\Sp{{\mathbb{S}}}
\def\T{\mathbb{T}}
\def\H{\mathbb{H}}
\def\hb{\hfil \break}
\def\ni{\noindent}
\def\i{\indent}
\def\a{\alpha}
\def\b{\beta}
\def\e{\epsilon}
\def\d{\delta}
\def\D{\Delta}
\def\G{\Gamma}
\def\g{\gamma}
\def\l{\lambda}
\def\m{\mu}
\def\s{\sigma}
\def\Si{\Sigma}
\def\th{\theta}
\def\z{\zeta}
\def\p{\phi}
\def\o{\omega}
\def\O{\Omega}
\def\t{\tau}
\def\L{\it \char'44}
\def\F{\mathcal{F}}
\def\B{\mathcal{B}}
\def\C{\mathcal{C}}
\def\half{\frac{1}{2}}
\def\qq{\qquad}
\def\ti{\tilde}
\ni m3f22soln10 \\
\begin{center}
{\bf M3F22 SOLUTIONS 10.  15.12.2017} 
\end{center}

\ni Q1.  {\it Exponential distributions and renewal}. \\
(i) Recall the exponential law $E(\l)$: density and distribution function 
$$
f(x) = \l e^{-\l x}, \qquad F(x) = 1 - e^{-\l x} \quad (x > 0).
$$
The Laplace transform of $f$ (Laplace-Stieltjes transform, or LST, of $F$) is
\begin{eqnarray*}
\hat F(s) 
&=& \int_{[0,\infty)} e^{-\l s} dF(x) \\
&=& \int_{[0,\infty)}\l  e^{-\l s}.e^{-\l x} dx \\
&=& \l/(\l + s).
\end{eqnarray*}
The LST of the $n$th convolution $F^{\ast n}$ of $F$ is the $n$th power of this.  Summing over $n$: for the renewal function
$$
U(x) = \sum_{n=0}^{\infty} F^{\ast n}(x),
$$
its LST is
$$
\hat U(s) 
= \sum_{n=0}^{\infty} (\l/(\l + s))^n 
= \frac{1}{\bigl( 1 - \l/(\l + s) \bigr)} 
= \frac{1}{s/(\l + s)} 
= (\l + s)/s: 
$$
$$
\hat U(s) = 1 + \l/s.
$$
Now with ${\delta}_0$ the Dirac measure at 0 (probability measure with all its mass 1 at the origin 0), its LST is 1.  The LST of Lebesgue measure on $(0,\infty)$ (the measure with mass $x$ on $(0,x)$) is, putting $u := sx$,
$$\int_0^{\infty} e^{-sx} dx = \int_)^{\infty} e^{-u} du/s = 1/s.
$$
Combining, for $F = E(\l)$,
$$
U(x) = {\delta}_0(x) + \l x \qquad (x \geq 0).
$$
{\it Interpretation}.  The first term 1 here just says that, by definition, there is always a renewal at time 0 (we always start with a new item).  After that, because the hazard rate for $E(\l)$ is the constant $\l$, the expected number of renewals in $(0,x)$ is $\l x$. \\
(ii) Recall that the mean $\mu$ of $E(\l)$ is, putting $u := \l x$ as above,
$$
\mu = \int_0^{\infty} x.f(x) dx = \int_0^{\infty} x.\l e^{-\l x} dx = \int_0^{\infty} u e^{-u} du/\l = 1/\l
$$
(the integral is $\Gamma(2) = 1! = 1$, or check by integration by parts).  So the Renewal Theorem holds:
$$
E(t) = E[N(t)] = 1 + t \l = 1 + t/\mu \sim t/\mu \quad (t \to \infty).
$$
 Blackwell's renewal theorem holds here with equality, as 
$$
U(x+h) - U(x) = h \l = h/\mu.
$$
The Key Renewal Theorem holds, as
$$
Z(t) = (z \ast U)(t) = \int_0^t z(u).u(t-u) du = \int_0^t z(u). \l \to \int_0^{\infty} z(u)du/\mu \quad (t \to \infty).
$$

\ni Q2.  {\it Gamma distributions and Renewal}. \\
(i) {\it Density}.
$$
\int f 
= \frac{1}{\G(\a)} \int_0^{\infty} e^{-\l x}.{\l}^{\a} x^{\a - 1} dx 
= \frac{1}{\G(\a)} \int_0^{\infty} e^{-u}. u^{\a - 1} du = 1,
$$
putting $u := \l x$ and using the definition of the Gamma function. \\
(ii) {\it Mean}.
\begin{eqnarray*}
\mu &=& \int x f(x) dx = \frac{1}{\G(\a)} \int_0^{\infty} x. e^{-\l x}.{\l}^{\a} x^{\a - 1} dx \\
&=& \frac{1}{\G(\a)} \int_0^{\infty} (u/\l). e^{-u}.u^{\a - 1} du 
= \frac{1}{\l \G(\a)} \int_0^{\infty} e^{-u}.u^{\a} du \\ 
&=& \frac{\G(\a + 1)}{\l \G(\a)} \\
&=& \a/\l.
\end{eqnarray*} 
(iii) {\it LST}: $\hat f(s)$. 
\begin{eqnarray*}
\hat f(s) 
&=& \frac{1}{\G(\a)} \int_0^{\infty} e^{-sx}.e^{-\l x}.{\l}^{\a} x^{\a - 1} dx   
= \frac{1}{\G(\a)} \int_0^{\infty} e^{-(\l + s)x}.{\l}^{\a} x^{\a - 1} dx \\
&=& = \frac{1}{\G(\a)} \int_0^{\infty} e^{-u} \Bigl( \frac{\l}{\l + s} \Bigr)^{\a} u^{\a - 1} du 
\quad (u := (\l + s)x) \\
&=& \Bigl( \frac{\l}{\l + s} \Bigr)^{\a}.
\end{eqnarray*}
(iv) {\it LST}: $\hat U(s)$. 
\begin{eqnarray*}
\hat U(s)
&=& \frac{1}{1 - \hat f(s)}
= \frac{1}{1 - \Bigl( \frac{\l}{\l + s} \Bigr)^{\a}} \\
&=& \frac{(\l + s)^{\a}}{(\l + s)^{\a} - {\l}^{\a}}
= \frac{(\l + s)^{\a}}{{\l}^{\a}\Bigl[ \Bigl( (1 + \frac{s}{\l} \Bigr)^{\a} - 1 \Bigr] }.
\end{eqnarray*}
(v) As $s \downarrow 0$,
$$
(\l + s)^{\a}/{\l}^{\a} \to 1, \qquad \Bigl[ \Bigl( (1 + \frac{s}{\l} \Bigr)^{\a} - 1 \Bigr] \sim \s.s/\l = \mu s,
$$
by the (generalised) Binomial Theorem (Newton: see M3H).  So
$$
\hat U(s) \sim \frac{1}{\mu s} \qquad (s \downarrow 0).
$$
(vi) So by HLK with $\rho = 1$,
$$
U(x) \sim x/\mu \qquad (x \to \infty),
$$
giving the Renewal Theorem in this case. \\ 

\hfil NHB \break


\end{document}