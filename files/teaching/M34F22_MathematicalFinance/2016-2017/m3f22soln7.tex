\documentclass[12pt]{article}
\usepackage{amsfonts}
\usepackage{amsmath}
\usepackage{pifont}
\usepackage{natbib}
\usepackage{color}
\def\bibfont{\footnotesize}
\setlength{\bibsep}{0pt plus 0.3ex}

\usepackage{url}

\begin{document}
\def\R{\mathbb{R}}
\def\C{\mathbb{C}}
\def\Z{\mathbb{Z}}
\def\N{\mathbb{N}}
\def\Q{\mathbb{Q}}
\def\D{\mathbb{D}}
\def\Sp{{\mathbb{S}}}
\def\T{\mathbb{T}}
\def\H{\mathbb{H}}
\def\hb{\hfil \break}
\def\ni{\noindent}
\def\i{\indent}
\def\a{\alpha}
\def\b{\beta}
\def\e{\epsilon}
\def\d{\delta}
\def\D{\Delta}
\def\G{\Gamma}
\def\g{\gamma}
\def\l{\lambda}
\def\m{\mu}
\def\s{\sigma}
\def\Si{\Sigma}
\def\th{\theta}
\def\z{\zeta}
\def\p{\phi}
\def\o{\omega}
\def\O{\Omega}
\def\t{\tau}
\def\L{\it \char'44}
\def\F{\mathcal{F}}
\def\B{\mathcal{B}}
\def\C{\mathcal{C}}
\def\half{\frac{1}{2}}
\def\qq{\qquad}
\def\ti{\tilde}
\ni m3f33soln7 \\
\begin{center}
{\bf M3F22 SOLUTIONS 7.  1.12.2017} 
\end{center}
\ni Q1. {\it Vega for calls}.   With $\phi(x) := e^{- \half x^2}/\sqrt{2 \pi}$, $\Phi(x) := \int_{-\infty}^x \phi(u) du$ the standard normal density and distribution functions, $\t := T-t$ the time to expiry, the Black-Scholes call price is
$$
C_t := S_t \Phi(d_1) - K e^{- r(T-t)} \Phi(d_2), \eqno(BS)
$$
$$
d_1 := \frac{\log(S/K) + (r + \half {\s}^2) \t}{\s \sqrt{\t}}, \qq
d_2 := \frac{\log(S/K) + (r - \half {\s}^2) \t}{\s \sqrt{\t}} = d_1 - \s \sqrt{\t}:
$$
$$
\phi(d_2) 
= \phi(d_1 - \s \sqrt{\t}) 
= \frac{e^{- \half (d_1 - \s \sqrt{\t})^2}}{\sqrt{2 \pi}} 
= \frac{e^{- \half d_1^2}}{\sqrt{2 \pi}}.e^{d_1 \s \sqrt{\t}}.e^{- \half {\s}^2 \t}:
$$
$$
\phi(d_2) = \phi(d_1).e^{d_1 \s \sqrt{\t}}.e^{- \half {\s}^2 \t}.
$$
Exponentiating the definition of $d_1$,
$$
e^{d_1 \s \sqrt{\t}} = (S/K).e^{r \t}.e^{\half {\s}^2 \t}.
$$
Combining,
$$
\phi(d_2) = \phi(d_1).(S/K).e^{r \t}: \qq K e^{-r \t} \phi(d_2) = S \phi(d_1). \eqno(\ast)
$$
Differentiating $(BS)$ partially w.r.t. $\s$ gives
$$
v := \partial C/\partial \s = S \phi(d_1) \partial d_1/\partial \s - K e^{-r \t} \phi(d_2) \partial d_2/\partial \s.
$$
So by $(\ast)$,
$$
v := \partial C/\partial \s = S \phi(d_1) \partial (d_1 - d_2)/\partial \s = S \phi(d_1) \partial (\s \sqrt{\t})/\partial \s = S \phi(d_1) \sqrt{\t} > 0.
$$
{\it Vega for puts}. \\
\i The same argument gives $v := \partial P/\partial \s > 0$, starting with the Black-Scholes formula for puts.  Equivalently, we can use put-call parity
$$
S + P - C = K e^{- r \t}: \qquad
\partial P/\partial \s = \partial C/\partial \s > 0.
$$
\ni {\it Interpretation}: "Options like volatility": the more uncertainty, i.e. the higher the volatility, the more the "insurance policy" of an option is worth.  So vega is positive for positions {\it long} in the option -- but negative for {\it short} positions. \\

\ni Q2.(i)  {\it Delta for calls}.
\begin{eqnarray*}
\Delta
&:=& \partial C/\partial S = \frac{\partial}{\partial S}[S \Phi(d_1) - K e^{-r \t} \Phi(d_2)] \\
&=& \Phi(d_1) + S \phi(d_1) \frac{\partial d_1}{\partial S} - K e^{-r \t} \phi(d_2) \frac{\partial d_2}{\partial S} \\
&=& \Phi(d_1) + S \phi(d_1) \frac{\partial (d_1 - d_2)}{\partial S},
\end{eqnarray*}
by Q1 $(\ast)$.  Since $d_1 - d_2 = \s \sqrt{\t}$ does not depend on $S$, this gives
$$
\Delta = \Phi(d_1) \in (0,1).
$$
Interpretation: the payoff $(S-K)_+$ is increasing in $S$, so the option price should be also -- and it is: $\Delta > 0$. \\
\i Also, $\Delta < 1$: options are to insure against adverse price movements.  This reflects that options are useful for this: if $\Delta$ were $\geq 1$, there would be no advantage in using options to hedge -- we would just use a combination of cash and stock. \\
(ii) {\it Delta for puts}.  Now put-call parity
$$
S + P - C = K e^{-r \t}
$$
and (i) give
$$
\partial P/\partial S =  \partial C/\partial S - 1  \in (-1,0).
$$
Interpretation: now the payoff $(K-S)_+$ is decreasing in $S$, so the option price should be also -- and it is.  That $\Delta > -1$ reflects that options are useful for insuring against adverse price movements (as above): if $\Delta$ were $\leq -1$, we would just use a combination of cash and stock.  \\

\ni Q3.  {\it Vega for American options}.  \\
\i The discounted value of an American option is the Snell envelope $\ti U_{n-1} = \max(\ti Z_{n-1}, E^*[\ti U_n \vert {\F}_{n-1}])$ of the discounted payoff $\ti Z_n$ (exercised early at time $n < N$), with terminal condition $U_N = Z_N, \ti U_N = \ti Z_N$.  As $\sigma$ increases, the $Z$-terms increase (vega is positive for European options).  As the $Zs$ increase, the $Us$ increase (above: backward induction on $n$ -- DP, as usual for American options).  Combining: as $\sigma$ increases, the $U$-terms increase.  So vega is also positive for American options. // \hfil NHB \break


\end{document}
