\documentclass[12pt]{article}
\usepackage{amsfonts}
\usepackage{amsmath}
\usepackage{pifont}
\usepackage{natbib}
\usepackage{color}
\def\bibfont{\footnotesize}
\setlength{\bibsep}{0pt plus 0.3ex}

\usepackage{url}

\begin{document}
\def\R{\mathbb{R}}
\def\C{\mathbb{C}}
\def\Z{\mathbb{Z}}
\def\N{\mathbb{N}}
\def\Q{\mathbb{Q}}
\def\D{\mathbb{D}}
\def\Sp{{\mathbb{S}}}
\def\T{\mathbb{T}}
\def\H{\mathbb{H}}
\def\hb{\hfil \break}
\def\ni{\noindent}
\def\i{\indent}
\def\a{\alpha}
\def\b{\beta}
\def\e{\epsilon}
\def\d{\delta}
\def\D{\Delta}
\def\G{\Gamma}
\def\g{\gamma}
\def\l{\lambda}
\def\m{\mu}
\def\s{\sigma}
\def\Si{\Sigma}
\def\th{\theta}
\def\z{\zeta}
\def\p{\phi}
\def\o{\omega}
\def\O{\Omega}
\def\t{\tau}
\def\L{\it \char'44}
\def\F{\mathcal{F}}
\def\B{\mathcal{B}}
\def\C{\mathcal{C}}
\def\half{\frac{1}{2}}
\ni m3f33soln6 \\
\begin{center}
{\bf M3F22 SOLUTIONS 6.  24.11.2017} 
\end{center}

\ni Q1 {\it Doubling strategy}.  (i) With $N$ the number of losses before the first win:
$$
P(N = k) = P(L, L, \cdots, L (k \hbox{ times}), W) =(\half)^k.\half = (\half)^{k+1}.
$$
That is, $N$ is geometrically distributed with parameter 1/2. As
$$
\sum_{k=0}^{\infty} P(N = k) = \sum_0^{\infty} (\half)^{k+1} = \half/(1 - \half) = 1,
$$
$P(N < \infty) = 1$: $N < \infty$ a.s.  So one is certain to win eventually. \\
(ii) Let $S_n$ be one's fortune at time $n$.  When $N = k$, one has losses at trials $1, 2, 3, \ldots, k$, with losses $1,2, 4, \ldots, 2^{k-1}$, followed by a win at trial $k+1$ (of $2^k$).  So one's fortune then is
$$
2^k - (1 + 2 + 2^2 + \ldots + 2^{k-1}) = 2^k - (2^k - 1) = 1,
$$
summing the finite geometric progression.  So one's eventual fortune is +1 (which, by (i), one is certain to win eventually). \\
(iii) $N$ has PGF
$$
P(s)
:= E[s^N]
= \sum_{n=0}^{\infty} s^k P(N = k)
= \sum_0^{\infty} s^k.(\half)^{k+1}
$$
$$
= \half \sum_0^{\infty} (\half s)^k = \half/(1 - \half s) = 1/(2 - s):
$$
$$
P'(s) = E[N s^{N-1}] = (2 - s)^{-2}; \qquad P'(1) = E[N] = 1.
$$
So the mean time the game lasts is 1. \\
(iv) As with the simple random walk (Q2 below): this is an impossible strategy to use in reality, for two reasons: \\
(a) It depends on one's opponent's cooperation.  What is to stop him trying this on you?  If he does, the game degenerates into a simple coin toss, with the winner walking away with a profit of 1 (pound, or million pounds, say) -- suicidally risky. \\
(b) Even with a cooperative opponent, it relies on the gambler having an unlimited amount of cash to bet with, or an unlimited line of credit -- both hopelesly unrealistic in practice. \\

\ni Q2 {\it First-passage time for simple random walk (SRW)}. \\
\i Let $F(s) := s^T = \sum_1^{\infty} P(T = n) s^n = \sum_1^{\infty} f_n s^n$ be the PGF of $T$ ($= T_1$, the first passage time to 1).  Since the first-passage time $T_2$ to 2 is the sum of the first-passage times from 0 to 1 (PGF $F$) and from 1 to 2 (PGF $F$ again), and these are independent (they involve disjoint blocks of independent tosses), $T_2$ has PGF $F_2(s) := E[s^{T_2}] = F(s)^2$. \\
\i Condition on the outcome $X_1$ of the first toss.  If this is head (+1), $T_1 = 1$.  If it is a tail ($-1$), $T = 1 + U$, where $U$, the first-passage time from $-1$ to 1, has PGF $F_2(s) = F(s)^2$ as above.  So
$$
F(s) := E[s^T]
= E[s^T | X_1 = +1] P(X_1 = +1) + E[s^T | X_1 = -1] P(X_1 = -1)
$$
$$
= \half.s + \half.s F(s)^2
$$
(as $1$ has PGF $s$).  So $F$ satisfies the quadratic
$$
\half s F(s)^2 - F(s) + \half s = 0. \qquad \hbox{So} \qquad F(s) = \frac{1 \pm \sqrt{1 - s^2}}{s}.
$$
We need to take the $-$ sign here (as $F(s)$ contains no $s^{-1}$ term):
$$
F(s) = \frac{1 - \sqrt{1 - s^2}}{s}.
$$
(i) Put $s = 1$: $F(1) = 1$, so $\sum_1^{\infty} P(T = n) = 1$, so $T < \infty$ a.s. \\
(ii)
$$
F'(s) = - \frac{1}{s^2} + \frac{\sqrt{1 - s^2}}{s} - \frac{1}{s}.\frac{\half (-2s)}{\sqrt{1 - s^2}}
= - \frac{1}{s^2} + \frac{\sqrt{1 - s^2}}{s} + \frac{1}{\sqrt{1 - s^2}}.
$$
So $F'(1) = E[T] = + \infty$. \\
(iii) In particular, $P(T = n) > 0$ for infinitely many $n$ (indeed, for all odd $n$).  So no bound can be put on our maximum net loss before we realise our eventual gain. \\
\i This strategy is even more unrealistic than that in Q1: it has all the disadvantages there, plus another -- infinite mean waiting time. \\

\hfil NHB \break


\end{document}