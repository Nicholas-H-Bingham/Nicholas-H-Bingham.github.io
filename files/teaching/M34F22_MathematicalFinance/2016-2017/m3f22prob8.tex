\documentclass[12pt]{article}
\usepackage{amsfonts}
\usepackage{amsmath}
\usepackage{pifont}
\usepackage{natbib}
\usepackage{color}
\def\bibfont{\footnotesize}
\setlength{\bibsep}{0pt plus 0.3ex}

\usepackage{url}

\begin{document}
\def\R{\mathbb{R}}
\def\C{\mathbb{C}}
\def\Z{\mathbb{Z}}
\def\N{\mathbb{N}}
\def\Q{\mathbb{Q}}
\def\D{\mathbb{D}}
\def\Sp{{\mathbb{S}}}
\def\T{\mathbb{T}}
\def\H{\mathbb{H}}
\def\hb{\hfil \break}
\def\ni{\noindent}
\def\i{\indent}
\def\a{\alpha}
\def\b{\beta}
\def\e{\epsilon}
\def\d{\delta}
\def\D{\Delta}
\def\G{\Gamma}
\def\g{\gamma}
\def\l{\lambda}
\def\m{\mu}
\def\s{\sigma}
\def\Si{\Sigma}
\def\th{\theta}
\def\z{\zeta}
\def\p{\partial}
\def\o{\omega}
\def\O{\Omega}
\def\t{\tau}
\def\L{\it \char'44}
\def\F{\mathcal{F}}
\def\B{\mathcal{B}}
\def\C{\mathcal{C}}
\def\half{\frac{1}{2}}
\ni m3f33prob8 \\
\begin{center}
{\bf M3F22 PROBLEMS 8.  1.12.2017} 
\end{center}

\ni Q1 ({\it The lognormal distribution}).  The {\it lognormal distribuktion} $LN(\mu, {\s}^2)$ is defined as the distribution of $X := e^Y$, where $Y \sim N(\mu, {\s}^2)$. v\\
(i) Show that
$$
E[X] = \exp \{ \mu + \half {\s}^2 \}.
$$
(ii) Explain why stock prices in the Black-Scholes model are lognormal. \\

\ni Q2 {\it Brownian covariance}.  The {\it covariance} of random variables $X$, $Y$ is
$$
cov(X,Y) := E[(X - E[X])(Y - E[Y])].
$$
Show that for $B = (B_t)$ Brownian motion (BM), its covariance is
$$
cov(B_s,B_t) = min(s,t).
$$
\i We quote that for a Gaussian process (one all of whose finite-dimensional distributions are Gaussian, such as BM), the process is characterised by its mean function and covariance function (so mean 0 and covariance $\min(s,t)$ characterise BM). \\

\ni Q3 {\it Brownian scaling}.  With $c > 0$ and $B$ Brownian motion, show that $B_c$, where
$$
B_c(t) := B(c^2t)/c,
$$
has the same covariance function $\min(s,t)$ as Brownian motion $B$.  Deduce that (as $B_c$ is also continuous and Gaussian) that $B_c$ {\it is} Brownian motion.  It is formed from $B$ by {\it Brownian scaling}. \\
\i Deduce that $B$ is {\it self-similar}: it reproduces itself it time and space are both scaled as above.  We call such a self-similar process a {\it fractal}. \\
\i If $Z$ is the zero-set of $B$ and $Z_c$ that of $B_c$, deduce that $Z, Z_c$ are fractals. \\
{\it Note}.  1.  Those with experience of computer graphics will recall `zooming in and blowing up' -- selecting a portion of a graphic (of particular interest), and blowing it up to full screen.  The essence of a fractal is that it {\it looks just the same} under this process -- even if we iterate it.  By contrast, a reasonably smooth function (differentiable, say) looks {\it quite different} -- it starts to look straight, {\it because it has a tangent}. \\
2.  You all know calculus, and you may not have met fractals before (or at least, not often).  You might suspect on this basis that `a typical function' is smooth (as in calculus), and that fractals are rare and pathological: the first examples of (what we now call) fractals were constructed to be continuous and nowhere differentiable (try drawing one!).  On the contrary: in a way that can be made precise (Baire category), {\it a typical continuous function is nowhere differentiable}. \\ 

\ni Q4 {\it Time inversion}.  For $B$ BM, and
$$
X_t := t B(1/t) \qquad (t \neq 0),
$$
$X = (X_t)$ is also BM. \\
\i Deduce or prove otherwise that for $B$ BM
$$
B(t)/t \to 0 \qquad (t \to \infty).
$$

\ni Q5.  By writing
$$
\int_0^t B(u) dB(u)
= \lim_{n \to \infty} \sum_{k=0}^{n-1} B(kt/n) (B((k+1)t/n)- B(kt/n))
$$
$$
=\sum \frac{1}{2}(B((k+1)t/n) + B(kt/n)).(B((k+1)t/n)- B(kt/n))
$$
$$
 - \sum \frac{1}{2}(B((k+1)t/n) - B(kt/n)).(B((k+1)t/n)- B(kt/n)),
$$
or otherwise, show that
$$
\int_0^t B(u) dB(u)
= \frac{1}{2}B(t)^2 - \frac{1}{2}t.
$$
Comment on the difference between this It\^o calculus result and ordinary (Newton-Leibniz) calculus. \\

\hfil NHB \break

\end{document}