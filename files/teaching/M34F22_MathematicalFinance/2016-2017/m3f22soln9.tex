\documentclass[12pt]{article}
\usepackage{amsfonts}
\usepackage{amsmath}
\usepackage{pifont}
\usepackage{natbib}
\usepackage{color}
\def\bibfont{\footnotesize}
\setlength{\bibsep}{0pt plus 0.3ex}

\usepackage{url}

\begin{document}
\def\R{\mathbb{R}}
\def\C{\mathbb{C}}
\def\Z{\mathbb{Z}}
\def\N{\mathbb{N}}
\def\Q{\mathbb{Q}}
\def\D{\mathbb{D}}
\def\Sp{{\mathbb{S}}}
\def\T{\mathbb{T}}
\def\H{\mathbb{H}}
\def\hb{\hfil \break}
\def\ni{\noindent}
\def\i{\indent}
\def\a{\alpha}
\def\b{\beta}
\def\e{\epsilon}
\def\d{\delta}
\def\D{\Delta}
\def\G{\Gamma}
\def\g{\gamma}
\def\l{\lambda}
\def\m{\mu}
\def\s{\sigma}
\def\Si{\Sigma}
\def\th{\theta}
\def\z{\zeta}
\def\p{\phi}
\def\o{\omega}
\def\O{\Omega}
\def\t{\tau}
\def\L{\it \char'44}
\def\F{\mathcal{F}}
\def\B{\mathcal{B}}
\def\C{\mathcal{C}}
\def\half{\frac{1}{2}}
\def\qq{\qquad}
\def\ti{\tilde}
\ni m3f22soln9 \\
\begin{center}
{\bf M3F22 SOLUTIONS 9.  15.12.2017} 
\end{center}

\ni Q1 {\it Gaussian distributions}.  By VI.2, if
${\bf \mu} \in {\R}^n$, ${\bf \Si}$ is a non-negative definite $n
\times n$ matrix, ${\bf X}$ has distribution $N({\bf \mu}, {\bf \Si})$
if it has characteristic function
$$
{\phi}_{\bf X} ({\bf t})
:= E \exp \{i {\bf t}^T .{\bf X} \}
= \exp \{ i {\bf t}^T .{\bf \mu} - \half {\bf t}^T {\bf \Si} {\bf t} \} \qquad ({\bf t} \in {\R}^n).
$$
Take ${\bf t} := t {\bf a}$ with $t$ real and ${\bf a}$ a constant $n$-vector:
$$
{\phi}_{{\bf a}^T {\bf X}} (t)
:= E \exp \{i t {\bf a}^T {\bf X} \}
= \exp \{ i t {\bf a}^T {\bf \mu} - \half t^2 {\bf a}^T {\bf \Si} {\bf a} \}.
$$
This says that
$$
{\bf a}^T {\bf X} \sim N({\bf a}^T {\bf \mu}, {\bf a}^T {\bf \Sigma} {\bf a}).
$$
\i For an It\^o integral $\int_0^t X_s dB_s$, we can approximate the integral by a sum.
In each term in the sum, the integrand can be treated like a constant (by continuity, the variation of the integrand over a small subinterval contributes only a second-order term, so is
negligible).  Each sum has a normal distribution, so a normal CF as above.  As the partition is refined, the sum tends to the integral in distribution.  So (L\'evy's continuity theorem) the
CFs converge to the CF of the limit.  So this limiting CF has the same functional form as a Gaussian CF.  So the integral is Gaussian: {\it Gaussianity is preserved under linear operations}. \\

\ni Q2 {\it (Ornstein-Uhlenbeck process}.  \\
(i) $V_t$ has mean $v_0 e^{-\b t}$, as
$$
E[ \int_0^t e^{\b u} dB_u] = \int_0^t e^{\b u} E[dB_u] = 0.
$$
By the It\^o isometry, $V_t$ has variance
$$
E[(\s e^{-\b t} \int_0^t e^{\b u} dB_u)^2] = {\s}^2 e^{-2 \b t} \int_0^t (e^{\b u})^2 du
$$
$$
= {\s}^2 e^{-2 \b t} \int_0^t e^{-2 \b u} du = {\s}^2 e^{-2 \b t} [e^{2 \b t} - 1]/(2 \b) = {\s}^2 [1 - e^{-2 \b t}]/(2 \b).
$$
%So the limit distribution as $t \to \infty$ is $N(0, {\s}^2/(2 \b))$. \\
(ii) For $u \geq 0$, the covariance is $cov(V_t, V_{t+u})$, which is
$$
{\s}^2 E[e^{-\b t} \int_0^t e^{\b v} dB_v.e^{-\b (t + u)} (\int_0^t + \int_t^{t+u}) e^{\b w} dB_w].
$$
By independence of Brownian increments, the $\int_t^{t+u}$ term contributes 0, leaving as before
$$
cov(V_t, V_{t+u}) = {\s}^2 e^{-\b u} [1 - e^{-2 \b t}]/(2 \b) \to {\s}^2 e^{-\b u}/(2 \b) \quad (t \to \infty).
$$
(iii)  The process $V$ is Markov (a diffusion), being the solution of the SDE $(OU)$.  $V$ is Gaussian, as it is obtained from the Gaussian process $B$ by linear operations (Q1). \\
(iv)  As $t \to \infty$, the mean $\to 0$.  The variance (take $u = 0$ in (ii)) $\to {\s}^2/(2 \b)$.  So the limit distribution is $N(0,{\s}^2/(2 \b))$:
$$
V_t \to N(0,{\s}^2/(2 \b)).
$$
{\it Note}. 1.  The parameter $\b$ governs how quickly the process `forgets its present position', and reverts towards its long-term mean.  It has the dimensions of inverse time;
$1/\b$ is called the {\it relaxation time}.\\
2. This limiting distribution $N(0, {\s}^2/(2 \b))$ is the {\it Maxwell-Boltzmann distribution} of Statistical Mechanics.  \\
3. The limit process is stationary Gaussian Markov, in contrast to Brownian motion, which is Gaussian Markov with stationary increments. \\

\i The $OU$ model is used for the theory of interest rates, because it shows {\it mean reversion}, as interest rates normally do; in this context it is called the {\it Vasicek model}. \\
\i Taking Bank Rate, say, as basic interest rate, this is usually a few percent, say in the range 3\% to 5\%, depending on the stage of the business cycle.  The Bank (of England, say) will
raise Bank Rate to slow the economy down to prevent overheating (and reduce inflation), and raise it to stimulate the economy during a recession.  But during the aftermath of the Credit Crunch,
or Crash of 2007 (US), 2008 (UK), ..., Bank Rate has stayed at historically extremely low levels (half a percent, or even less) for a long period -- unprecedented in the history of the Bank of England since its foundation in 1694.  Clearly the $OU$ model does not apply well to the current situation (interest rates are {\it negative} in some countries!).  This is not surprising, as the current situation is unprecedented.  \hfil NHB \break



\end{document}