\documentclass[12pt]{article}
\usepackage{amsfonts}
\begin{document}
\def\R{\mathbb{R}}
%\def\C{\mathbb{C}}
\def\Z{\mathbb{Z}}
\def\N{\mathbb{N}}
\def\Q{\mathbb{Q}}
%\def\D{\mathbb{D}}
\def\T{\mathbb{T}}
\def\hb{\hfil \break}
\def\ni{\noindent}
\def\i{\indent}
\def\a{\alpha}
\def\b{\beta}
\def\e{\epsilon}
\def\d{\delta}
\def\D{\Delta}
\def\g{\gamma}
\def\qq{\qquad}
\def\L{\Lambda}
\def\E{\cal E}
\def\G{\Gamma}
\def\F{\cal F}
\def\K{\cal K}
%\def\O{\cal O}
\def\A{\cal A}
\def\B{\cal B}
\def\C{\cal C}
%\def\S{\cal S}
\def\M{\cal M}
\def\P{\cal P}
\def\Om{\Omega}
\def\om{\omega}
\def\s{\sigma}
\def\t{\theta}
\def\th{\theta}
\def\Th{\Theta}
\def\z{\zeta}
\def\p{\phi}
\def\m{\mu}
\def\n{\nu}
\def\l{\lambda}
\def\Si{\Sigma}
\def\q{\quad}
\def\qq{\qquad}
\def\half{\frac{1}{2}}
\def\hb{\hfil \break}
\def\half{\frac{1}{2}}
\def\pa{\partial}
\def\r{\rho}
\def\hb{\hfil \break}
\def\ni{\noindent}
\def\i{\indent}
\def\ti{\tilde}
\ni m3f22soln1.tex \\
\begin{center}
{\bf SOLUTIONS 1.  14.10.2016}
\end{center}

\ni Q1 ({\it Compound Poisson: CF, mean and variance}). \\
(i) The characteristic function (CF) follows from 
\begin{eqnarray*}
\psi(t) = E[e^{itY}]
&=& E[\exp \{it (X_1 + \ldots + X_N) \}] \\
&=& \sum_n E[\exp \{it(X_1 + \ldots + X_N) \}|N = n].P(N = n) \\
&=& \sum_n e^{-\l} {\l}^n/n!.E[\exp \{it(X_1 + \ldots + X_n) \}] \\
&=& \sum_n e^{-\l} {\l}^n/n!.(E[\exp \{it X_1 \}])^n 
= \sum_n e^{-\l} {\l}^n/n!.\phi(t)^n \\
&=& \exp \{-\l (1 - \phi(t)) \}.
\end{eqnarray*}
(ii) If $X$ has CF $\phi(t) = E[e^{iXt}]$: differentiating,
$$
{\phi}'(t) = E[i X e^{iXt}]: \qquad {\phi}'(0) = i E[X]; \qquad E[X] = - i {\phi}'(0);
$$
$$
{\phi}''(t) = E[- X^2 e^{iXt}]: \qquad {\phi}''(0) = - E[X^2]; \qquad E[X^2] = - {\phi}''(0).
$$  
Differentiate the CF $\psi$ of $Y$:
$$
{\psi}'(t) = \psi(t).\l {\phi}'(t);
\qquad
{\psi}''(t) = {\psi}'(t).\l {\phi}'(t) + \psi(t).\l {\phi}''(t).
$$
By above, ($\phi(0)
= 1$ and) ${\phi}'(0) = i \m$, ${\phi}''(0) = -E[X^2]$,
$$
{\psi}'(0) = \l {\phi}'(0) = \l.i \m,
$$
and as also ${\psi}'(0) = i EY$, this gives
$$
E[Y] = \l \m.
$$
Thus the mean of the random sum $Y := X_1 + \cdots + X_N$ is the product of the means of $X$ (short for a typical $X_i$) and $N$:
$$
E[Y] := E[X_1 + \cdots + X_N] = E[X].E[N].
$$
This is (a special case of) {\it Wald's identity} (Abraham Wald (1902-1950) in 1944).  Similarly,
$$
{\psi}''(0) = i \l \m.i \l \m + \l {\phi}''(0) = - {\l}^2 {\m}^2 -
\l E[X^2],
$$
and also ($\psi(0) = 1$, ${\psi}'(0) = i \lambda \mu$ and)
${\psi}''(0) = -E[Y^2]$. So
$$
var \ Y = E[Y^2] - [EY]^2 = {\l}^2 {\m}^2 + \l E[X^2] - {\l}^2 {\m}^2 = \l E[X^2].
$$
(ii) Apply (i): $N_t$ has mean $\l t$ and variance $\lambda t$. // \\

\ni Q2 ({\it Normal MGF}).
$$
M_X(t) = \int_{-\infty}^{\infty} e^{tx} f_X(x) dx = \int_{-\infty}^{\infty} e^{tx}.\frac{1}{\sqrt{2 \pi} \s} \exp \{-\frac{1}{2}(x - \mu)^2/{\s}^2 \} dx.
$$
Make the substitution $u := (x - \mu)/\s$: $x = \mu + \s u$, $dx = \s du$:
$$
M_X(t) = \int_{-\infty}^{\infty} e^{t(\mu + \s u)}.\frac{1}{\sqrt{2 \pi}} \exp \{-\frac{1}{2}u^2 \} du
= e^{\mu t}.\int_{-\infty}^{\infty} e^{\s t u}.\frac{1}{\sqrt{2 \pi}} \exp \{-\frac{1}{2}u^2 \} du.
$$
Completing the square in the exponent on the right,
$$
M(t)
= e^{\mu t}.\int_{-\infty}^{\infty} \frac{1}{\sqrt{2 \pi}} \exp \{-\frac{1}{2}[u^2 - 2\s t u] \} du
$$
$$
= e^{\mu t}.\int_{-\infty}^{\infty} \frac{1}{\sqrt{2 \pi}} \exp \{-\frac{1}{2}[(u - \s t)^2 - {\s}^2 t^2] \} du
= e^{\mu t + \frac{1}{2}{\s}^2 t^2}.\int_{-\infty}^{\infty} \frac{1}{\sqrt{2 \pi}} \exp \{-\frac{1}{2}(u - \s t)^2 \} du.
$$
The integral on the right is 1 (a density integrates to 1 -- of $N(\s t,1)$ as it stands, or of $N(0,1)$ after the substitution $v := u - \s t$), giving
$$
M(t) = \exp \{ \mu t + \frac{1}{2}{\s}^2 t^2 \}.
$$

\ni Q3 ({\it Normal CF}).  In Q2, $t$ is real, but if we formally replace $t$ by $it$, we get the normal CF as
$$
E[e^{itX}] = \exp \{ i \mu t - \frac{1}{2}{\s}^2 t^2 \}.
$$
This is indeed correct, but a formal proof needs some Complex Analysis.  There are two ways to see this: \\
(i) {\it Analytic continuation}.  \\
\i If we let $t$ in Q1 be {\it complex}, the MGF $\exp \{ \mu t + \frac{1}{2}{\s}^2 t^2 \}$ becomes an analytic (= holomorphic) function with no singularities in the whole complex $t$-plane $\C$ -- that is, an {\it entire} (= integral) function.  For entire functions, `what looks right, is right', by {\it analytic continuation} (similarly for analytic functions, within domains of analyticity).  For background, see any decent book on Complex Analysis, or e.g. my home-page, M2P3 Complex Analysis link, 2011 L 22 - 23.  The technique is very powerful, and well worth mastering. \\
(ii) {\it Cauchy's (Residue) Theorem}. \\
\i  Alternatively, one can prove this by integrating the function $e^{\half z^2}$ round a long thin rectangle in the complex $z$-plane, and using Cauchy's Theorem -- or, Cauchy's Residue Theorem (actually, there are no residues, as there are no singularities -- as above).  See e.g. M2P3 L 26 - 27. \\

\hfil NHB \break

\end{document}









(b)  Given $N$, $Y = X_1 + \ldots + X_N$ has mean $N EX = N \m$
and variance $N \ var \ X = N{\s}^2$.  As $N$ is Poisson with
parameter $\l$, $N$ has mean $\l$ and variance $\l$.  So by the
Conditional Mean Formula,
$$
EY = E[E(Y|N)] = E[N \m] = \l \m.
$$
By the Conditional Variance Formula,
\begin{eqnarray*}
var \ Y 
&=& E[var(Y|N)] + var \ E[Y|N]  \\
&=& E[N var \ X] + var([N \ E[X])  \\
&=& E[N].var \ X + var \ N.(EX)^2 \\
&=& \l [E[X^2] - (E[X])^2] + \l.(E[X])^2 \\
&=& \l E[X^2] = \l ({\sigma}^2 + {\mu}^2).  
\end{eqnarray*}







\ni Q1.  {\it Keynes and Hajek/Friedman}. \\
\ni {\it Keynes} \\
W: "John Maynard Keynes, 1st Baron Keynes, CB, FBA (`KAYNZ'; 5 June 1883 – 21 April 1946) was a British economist whose ideas have fundamentally affected the theory and practice of modern macroeconomics, and informed the economic policies of governments. He built on and greatly refined earlier work on the causes of business cycles, and is widely considered to be one of the founders of modern macroeconomics and the most influential economist of the 20th century. His ideas are the basis for the school of thought known as Keynesian economics, and its various offshoots. \\
\i In the 1930s, Keynes spearheaded a revolution in economic thinking, overturning the older ideas of neoclassical economics that held that free markets would, in the short to medium term, automatically provide full employment, as long as workers were flexible in their wage demands. Keynes instead argued that aggregate demand determined the overall level of economic activity, and that inadequate aggregate demand could lead to prolonged periods of high unemployment. According to Keynesian economics, state intervention was necessary to moderate "boom and bust" cycles of economic activity. He advocated the use of fiscal and monetary measures to mitigate the adverse effects of economic recessions and depressions. Following the outbreak of World War II, Keynes's ideas concerning economic policy were adopted by leading Western economies. In 1942, Keynes was awarded a hereditary peerage ... Keynes died in 1946, but during the 1950s and 1960s the success of Keynesian economics resulted in almost all capitalist governments adopting its policy recommendations. \\
\i Keynes's influence waned in the 1970s, partly as a result of problems that began to afflict the Anglo-American economies from the start of the decade, and partly because of critiques from Milton Friedman and other economists who were pessimistic about the ability of governments to regulate the business cycle with fiscal policy. However, the advent of the global financial crisis of 2007 -- 08 caused a resurgence in Keynesian thought. Keynesian economics provided the theoretical underpinning for economic policies undertaken in response to the crisis by President George W. Bush of the US, PM Gordon Brown of the UK, and other heads of governments. \\
\i In 1999, Time magazine included Keynes in their list of the 100 most important and influential people of the 20th century, commenting that: "His radical idea that governments should spend money they don't have may have saved capitalism." He has been described by The Economist as "Britain's most famous 20th-century economist." ... " \\
\ni {\it Hayek} \\
W: "Friedrich Hayek CH (8 May 1899 – 23 March 1992), born in Austria-Hungary as Friedrich August von Hayek and frequently referred to as F. A. Hayek, was an Austrian, later British, economist and philosopher best known for his defence of classical liberalism. Hayek shared the Nobel Memorial Prize in Economic Sciences (with Gunnar Myrdal) for his "pioneering work in the theory of money and economic fluctuations and ... penetrating analysis of the interdependence of economic, social and institutional phenomena". \\
\i Hayek was a major political thinker of the twentieth century, and his account of how changing prices communicate information which enables individuals to co-ordinate their plans is widely regarded as an important achievement in economics. \\
\i Hayek served in World War I and said that his experience in the war and his desire to help avoid the mistakes that had led to the war led him to his career. Hayek lived in Austria, Great Britain, the United States and Germany, and became a British subject in 1938. He spent most of his academic life at the LSE, the University of Chicago, and the University of Freiburg. \\
\i In 1984, he was appointed CH by Queen Elizabeth II on the advice of Prime Minister Margaret Thatcher for his "services to the study of economics" ...." \\
\ni {\it Friedman} \\
W: "Milton Friedman (July 31, 1912 – November 16, 2006) was an American economist, statistician, and writer who taught at the University of Chicago for more than three decades. He was a recipient of the 1976 Nobel Prize in Economic Sciences, and is known for his research on consumption analysis, monetary history and theory, and the complexity of stabilization policy. As a leader of the Chicago school of economics, he profoundly influenced the research agenda of the economics profession. A survey of economists ranked Friedman as the second most popular economist of the twentieth century after John Maynard Keynes, and The Economist described him as "the most influential economist of the second half of the 20th century ... possibly of all of it." \\
\i Friedman's challenges to what he later called "naive Keynesian" (as opposed to New Keynesian) theory began with his 1950s reinterpretation of the consumption function, and he became the main advocate opposing Keynesian government policies. In the late 1960s, he described his own approach (along with all of mainstream economics) as using "Keynesian language and apparatus" yet rejecting its "initial" conclusions. \\
\i During the 1960s, he promoted an alternative macroeconomic policy known as "monetarism". He theorized there existed a "natural" rate of unemployment and argued that governments could increase employment above this rate, e.g., by increasing aggregate demand, only at the risk of causing inflation to accelerate. He argued that the Phillips curve was not stable and predicted what would come to be known as stagflation. Though opposed to the existence of the Federal Reserve System, Friedman argued that, given that it does exist, a steady, small expansion of the money supply was the only wise policy. \\
\i Friedman was an economic adviser to Republican U.S. President Ronald Reagan. His political philosophy extolled the virtues of a free market economic system with minimal intervention. He once stated that his role in eliminating U.S. conscription was his proudest accomplishment .... In his 1962 book {\sl Capitalism and Freedom}, Friedman advocated policies such as a volunteer military, freely floating exchange rates, abolition of medical licenses, a negative income tax, and school vouchers. His ideas concerning monetary policy, taxation, privatization and deregulation influenced government policies, especially during the 1980s. His monetary theory influenced the Fed's response to the global financial crisis of 2007 –- 08 ... \\
\i Milton Friedman's works include many monographs, books, ...  His books and essays were widely read, and have had an international influence, including in former Communist states." \\

\ni {\it Influence on policy and events}. \\
\i The crucial events of the 20th C. were WWI (1914-18) and WWII (1939-45).  WWI led to the collapse of four empires (Russian, German, Austro-Hungarian, Ottoman), the rise of communism (c. 1917 -- c. 1991), the Depression (or Slump) of 1929 on, Roosevelt and the New Deal in the US and the rise of fascism/nazism in Europe, pre-WWII.  The new post-WWII institutions (UN, World Bank, International Monetary Fund, etc.) brought about the post-war consensus, 1945 -- 1979/80 (and the collapse of the remaining empires: British, French and Dutch).  {\it The dominant economic thinking during the post-war consensus was Keynesian.}  Following the coming to power of {\it Thatcher} (UK, 1979) and {\it Reagan} (US, 1980), this was largely replaced by a {\it neo-con} (`con for conservative') consensus, influenced by Friedman, and (indirectly) Hayek.  The financial crisis of 2007 (US)/2008 (UK) -- ongoing -- has led to major re-thinking all round. \\
\i You should be broadly aware of the above, at least on outline, for background. \\

\hfil NHB \break

\end{document}

