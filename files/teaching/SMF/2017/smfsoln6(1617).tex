\documentclass[12pt]{article}
\begin{document}
\def\ni{\noindent}
\def\i{\indent}
\def\a{\alpha}
\def\b{\beta}
\def\e{\epsilon}
\def\d{\delta}
\def\g{\gamma}
\def\qq{\qquad}
\def\q{\quad}
\def\L{\Lambda}
\def\C{\cal C}
\def\E{\cal E}
\def\G{\Gamma}
\def\F{\cal F}
\def\K{\cal K}
\def\O{\cal O}
\def\A{\cal A}
\def\B{\cal B}
\def\S{\cal S}
\def\N{\cal N}
\def\M{\cal M}
\def\P{\cal P}
\def\Om{\Omega}
\def\om{\omega}
\def\s{\sigma}
\def\t{\theta}
\def\z{\zeta}
\def\p{\phi}
\def\m{\mu}
\def\n{\nu}
\def\b{\beta}
\def\e{\epsilon}
\def\l{\lambda}
\def\Si{\Sigma}
\def\half{\frac{1}{2}}
\def\hb{\hfil \break}
% \ni smfsoln6(17).tex \\
\begin{center}
{\bf SMF SOLUTIONS 6.  2.3.2017} \\
\end{center}

\ni Q1 ({\it Brownian bridge}). \\
\i With the Brownian bridge defined as
$$
B_0(t) :- B(t) - t B(1),
$$
the mean is $E[B_0(t)] = E[B(t)] - t E[B(1)] = 0$.  So the covariance is, for $s, t \in [0,1]$    (as Brownian motion $B$ has covariance $cov(B(s), B(t)) = E[B(s)B(t)] = \min (s,t)$)
\begin{eqnarray*}
cov(B_0(s), B_0(t))
&=& E[B_0(s).B_0(t)] \\
&=& E[(B(s) - s B(1))(B(t) - t B(1)] \\
&=& E[B(s) B(t)] - t E[B(s) B(1)] - s E[B(t) B(1)] + st E[B(1)^2] \\
&=& \min (s,t) - st - st + st \\
&=& \min (s,t) - st.
\end{eqnarray*}

\ni Q2 ({\it Median; breakdown point}). \\
\i For simplicity, take the sample size odd.  The median is the point with half the data points below it.  These can go off to $- \infty$ (and/or the points above can go off to $+ \infty$) without dragging the median with them; but if more than half the points do this, they will drag the median with them.  So, the median has breakdown point $1/2$, as stated. \\

\ni Q3 ({\it Quartiles; semi-inter-quartile range}, $SIQ$). \\
\i The lower quartile has a quarter of the data points beneath it.  These can go off to $- \infty$ without dragging the lower quartile with them; but if more than a quarter do this, they will drag the lower quartile with them.  So, the lower quartile has breakdown point 1/4.  Similarly, so does the upper quartile.  So the semi-interquartile range $SIQ$ (half their difference) also has breakdown point 1/4. \\ 

\hfil NHB \break

\end{document}