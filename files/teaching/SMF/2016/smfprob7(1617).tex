%smfprob6.tex 12.4.2012 Was smfprob3.tex 23.12.2010
\documentclass[12pt]{article}
\begin{document}
\def\ni{\noindent}
\def\i{\indent}
\def\a{\alpha}
\def\b{\beta}
\def\e{\epsilon}
\def\d{\delta}
\def\g{\gamma}
\def\qq{\qquad}
\def\q{\quad}
\def\L{\Lambda}
\def\C{\cal C}
\def\E{\cal E}
\def\G{\Gamma}
\def\F{\cal F}
\def\K{\cal K}
\def\O{\cal O}
\def\A{\cal A}
\def\B{\cal B}
\def\S{\cal S}
\def\N{\cal N}
\def\M{\cal M}
\def\P{\cal P}
\def\Om{\Omega}
\def\om{\omega}
\def\s{\sigma}
\def\t{\theta}
\def\z{\zeta}
\def\p{\phi}
\def\m{\mu}
\def\n{\nu}
\def\b{\beta}
\def\e{\epsilon}
\def\l{\lambda}
\def\Si{\Sigma}
\def\half{\frac{1}{2}}
\def\hb{\hfil \break}
\ni smfprob7(1617).tex \\
\begin{center}
{\bf SMF PROBLEMS 7.  2.3.2017} \\
\end{center}

\ni Q1. (i) For a Bernoulli distribution $B(p)$ with uniform prior on $p \in [0,1]$, show that the posterior distribution is a Beta distribution, and find the parameters. \\
(ii)  Repeat with a Beta prior $B(\a, \b)$. \\

\ni Q2. For the Bernoulli distribution $B(p)$, find \\
(i) the information per reading; \\
(ii) the Jeffreys prior. \\

\ni Q3.  Find the mean of $B(\a, \b)$. \\

\ni Q4.  Hence find the posterior mean in Q1(ii), and interpret this as the sample size $n$ increases. \\

\ni Q5 ({\it Convolutions of Gammas and Euler's integral for the Beta function}).  Write $f_{\a}$ for the exponential density with parameter ${\a}$:
$$
f_{\a}(x) = x^{\a - 1} e^{-x}/\G(\a) \qquad (x > 0).
$$
(i) Show that
$$f_{\a} \ast f_{\b} = f_{\a + \b}.
$$
(ii)  Deduce Euler's integral for the Beta function:
$$
B(\a, \b) := \int_0^1 x^{\a - 1} (1-x)^{\b - 1} dx = \frac{\G(\a) \G(\b)}{\G(\a + \b)}.
$$

\ni Q6.  For the shifted exponential distribution with parameter $\t > 0$ (density $e^{-(x-\t)}$ for $x > \t$), \\
(i) find the MLE; \\
(ii) find for each $n$ a sufficient statistic; \\
(iii) if $\t$ has prior density $\l e^{-\l \t}$ ($\t \sim E(\l)$), find the posterior density to within a multiplicative constant.\\

\hfil NHB \break

\end{document}

