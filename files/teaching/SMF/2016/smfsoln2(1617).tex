\documentclass[12pt]{article}
\usepackage{amsfonts}
\begin{document}
\def\R{\mathbb{R}}
\def\C{\mathbb{C}}
\def\Z{\mathbb{Z}}
\def\N{\mathbb{N}}
\def\Q{\mathbb{Q}}
\def\D{\mathbb{D}}
\def\T{\mathbb{T}}
\def\hb{\hfil \break}
\def\ni{\noindent}
\def\i{\indent}
\def\a{\alpha}
\def\b{\beta}
\def\e{\epsilon}
\def\d{\delta}
\def\De{\Delta}
\def\g{\gamma}
\def\qq{\qquad}
\def\L{\Lambda}
\def\E{\cal E}
\def\G{\Gamma}
\def\F{\cal F}
\def\K{\cal K}
%\def\O{\cal O}
\def\A{\cal A}
\def\B{\cal B}
%\def\S{\cal S}
\def\M{\cal M}
\def\P{\cal P}
\def\Om{\Omega}
\def\om{\omega}
\def\s{\sigma}
\def\t{\theta}
\def\th{\theta}
\def\Th{\Theta}
\def\z{\zeta}
\def\p{\phi}
\def\m{\mu}
\def\n{\nu}
\def\l{\lambda}
\def\Si{\Sigma}
\def\q{\quad}
\def\qq{\qquad}
\def\half{\frac{1}{2}}
\def\hb{\hfil \break}
\def\half{\frac{1}{2}}
\def\pa{\partial}
\def\r{\rho}
\def\hb{\hfil \break}
\def\ni{\noindent}
\def\i{\indent}
\ni smfsoln2(1617)\\
\begin{center}
{\bf SMF SOLUTIONS 2. 2.2.2017}
\end{center}

\ni Q1.  (i) Consider
$$
Q(\l) := \int (\l f + g)^2 = {\l}^2 \int f^2 + 2 \l \int fg + \int g^2.
$$
This is always non-negative, so its discriminant is $\leq 0$ (if it were $> 0$, there would be two distinct real roots with a sign change between them).  So ("$b^2 - 4ac \leq 0$")
$$
(\int fg)^2 \leq \int f^2 \int g^2.
$$
Equality holds iff $Q$ has a double root, ${\l}_0$ say.  Then
$$
Q({\l}_0) = \int ({\l}_0 f + g)^2 = 0.
$$
This forces
$$
{\l}_0 f + g = 0
$$
(a.e.), the required linear relation. \\
(ii) We apply this to $X - E[X]$, $Y - E[Y]$, we find equality in CR iff a linear relation of the form $a(X - E[X]) + b(Y - E[Y]) = 0$ holds.  Taking $X = {\ell}'$ (recall $E[{\ell}'] = 0$), $Y = u$ (recall $EY = \t$ as $u$ is unbiased for $\t$), we find
$$
u - \t = a {\ell}'
$$
for some $a$ (so $b = \t$). \\
(iii) So
$$
{\ell}' = u/a(\t) - \t/a(\t): \qquad \ell = u \int d\t/a(\t) - \int \t d\t/a(\t) + k({\bf X}):
$$
$$
L(\t; {\bf X}) = \exp \{ \a(\t) u({\bf X}) + \b(\t) + k({\bf X}) \}.
$$
(iv) By Fisher-Neyman, this shows $u$ is sufficient for $\t$; by Lehmann-Scheff\'e, this shows $u$ is minimal sufficient for $\t$. \\

\ni Q2.
$$
\ell = -n \log 2 - \sum |x_i - \t|.
$$
To maximise this -- i.e. minimise $\sum |x_i - \t|$ -- draw a graph.  From this, the sum is minimised by $\t = Med$, and increases linearly on either side of the sample median.  So the MLE is
$\hat{\m} = Med$. \\
(ii) With one reading, as above, $\ell$ decreases with slope -1 to the right of $Med$, slope +1 to the left of $Med$.  So $({\ell}')^2 = 1$ (except at $\l = Med$, where the derivative is not defined -- but we are going to integrate, and so can neglect null sets, let alone single points, so this does not matter).  So $I = \int (\partial \log f/\partial \t)^2 f = \int f = 1$, as $f$ is a
density.  So the CR bound is $1/n$. \\
\i  We are given that $Med$ is asymptotically normal, and that its mean is $med = \t$, so $Med$ is asymptotically unbiased. By symmetry, the population median is $med = \t$, where the density is $\half$.  So $4 f(med)^2 = 1$, and the asymptotic variance of the sample median is $1/n$, the CR bound, so $Med$ is also asymptotically efficient.\\

\ni Q3. (i)
$$
f(x; \m) = \frac{1}{\pi(1 + (x - \m)^2)}, \qquad \ell = \log f = c - \log[1 + (x - \m)^2],
$$
$$
{\ell}' = \frac{2(x - \m)}{1 + (x - \m)^2}, \qquad {\ell}'({\bf x}; \t) = 2 \sum_1^n \frac{(x_i - \m)}{1 + (x_i - \m)^2}.
$$
But (Q1) we have efficiency iff ${\ell}'$ factorises in the form ${\ell}'({\bf x}; \t) = A(\t) (u({\bf x}) - \t)$.  The likelihood here does not factorise, so there is no efficient estimator. \\
(ii) The information per reading is
$$
E[({\ell}')^2] = \int (\partial f/\partial \m)^2 f
= \frac{4}{\pi} \int \frac{(x - \m)^2}{[1 + (x - \m)^2]^3} dx
= \frac{4}{\pi} \int \frac{x^2}{[1 + x^2]^3} dx = \frac{4}{\pi} I,
$$
say.  One can evaluate $I$ by Complex Analysis ($f(z) := z^2/[1 + z^2]^3$, round the contour $\Gamma$ -- semicircle in the upper half-plane on base $[-R,R]$; $f$ has a triple pole inside $\Gamma$ of residue $-i/16$, so $I = 2 \pi i \ Res = \pi/8$), giving the information per reading as $\half$.  So the information in a sample of size $n$ is $n/2$, and the MLE has asymptotic variance $2/n$.  As in Q2, the sample median has asymptotic variance ${\pi}^2/4n$.  So the asymptotic efficiency is their ratio, $8/{\pi}^2 \sim 81 \%$.  \hfil NHB \break



\end{document} 