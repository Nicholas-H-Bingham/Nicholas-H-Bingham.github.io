%m2pm3l11.tex 7.2.2010 From Ingolf Becker's notes to Lectures 1-10, 2.2.2010.
\documentclass[12pt]{article}
\begin{document}
\def\ni{\noindent}
\def\i{\indent}
\def\a{\alpha}
\def\e{\epsilon}
\def\d{\delta}
\def\G{\Gamma}
\def\s{\sigma}
\def\t{\theta}
\def\z{\zeta}
\def\p{\partial}
\ni m2pm3l10.tex \\

\ni {\bf Lecture 11}. 4.2.2010.\\

\ni 4. {\it Hyperbolic functions}\\
    \[\begin{array}{rclrcl}
    \cosh z&=&\frac{1}{2}(e^z+e^{-z}),&\frac{d}{dz}\cosh z&=&\sinh z,\\
    \sinh z&=&\frac{1}{2}(e^z-e^{-z}),&\frac{d}{dz}\cosh z&=&\cosh z,\\
    \tanh z&=&\frac{\sinh z}{\cosh z},&\cosh^2 z-\sinh^2 z&=&1.
    \end{array}\]

\ni 5. {\it Logarithms}\\
\i Recall that in Real Analysis, log is the inverse function of
$\exp$.
    \[\log x=y \textrm{ means } e^y=x\]
This extends to ${\bf C}$, as follows for $z,w\in {\bf C}$,
    \[\log z=w \textrm{ means } e^w=z\]
\emph{But:} $e^{2\pi i}=1$, so $e^{2\pi ki}=1\;\forall k\in {\bf Z}$. So if $e^w=z$, also
$e^{w+2\pi k i}=z$. So if $\log z=w$, also $\log z=w+2\pi ki$: the log is \emph{not}
single-valued, and is determined only to within additive multiples of $2\pi i$. In
particular, $\log$ is not a \emph{function} as previously defined.\\
\i There are three ways to proceed:\\
(i) {\it Many-valued functions}.\\
We can regard $\log$ as a \emph{many-valued function} (as with
$\sin^{-1}=\arcsin$).\\
(ii) {\it Cuts}.\\
\emph{Cut} the complex plane ${\bf C}$ by removing (e.g.) the
negative real axis.\\
(iii) {\it Riemann surfaces}\\
Think of $\log$, not going from ${\bf C}$ to ${\bf C}$, but from
$R$ to ${\bf C}$, where $R$ is a doubly infinite stack of copies
${\bf C}_k$ of ${\bf C}$, one for each $k\in{\bf Z}$, `spliced
together' along their positive real axes so that
$\theta\rightarrow\theta+2\pi$ takes one from ${\bf C}_k$ to ${\bf
C }_{k+1}$. This is a \emph{Riemann surfaces} (G.F.B. RIEMANN
(1926-66) in 1851; Felix KLEIN (1849-1925) in 1882).\\
\i The origin O is a `point of bad behaviour' of $\log z$: a
\emph{singularity} (see later) - a branch-point.
\[\begin{array}{rrcl}
\textrm{If }e^{w_i}=z_i \quad (i=1,2): & e^{w_1+w_2}&=&e^{w_1}e^{w_2}=z_1\cdot z_2,\\
&w_1+w_2&=&\log(z_1\cdot z_2),\\
&\log(z_1\cdot v_2)&=&\log z_1 + \log z_2,\\
&\log\left(\frac{z_1}{z_2}\right)&=&\log z_1 - \log z_2.
\end{array}\]
\i Recall: in the real case, $y=\log x$ if $e^y=x$.
Differentiating (implicitly) w.r.t. $x$: $e^y dy/dx=1$, $dy/dx=1/{e^y}=1/x$, $dy=dx/x$,
$y=\int dx/x$.\\
As $e^0=0$, $\log 1=0$: $y=\int_1^x du/u$; and $y=\log x$: $\log x =\int_1^x du/u$.\\
Using complex differentiation (II.2), and complex integration (II.4), we can extend this to
${\bf C}$, \emph{provided}:\\
(i) We integrate along the \emph{line-segment} $[1,z]$ joining $1$ to $z$ in ${\bf C}$;\\
(ii) $[1,z]$ \emph{avoids} the singularity $z=0$ (\emph{branch-point}).\\
So in ${\bf C}$, $\log z=\int_1^z dw/w$ or
$\int_{[1,z]} dw/w$ works, \emph{provided} that $z$ does
not lie on the negative real line or O, i.e. \emph{provided} that
we work with the \emph{cut} plane. For details, see Exam, 2009, Q2.\\

\ni 6. {\it Complex Powers}\\
Recall in the real case, $a^x=e^{x\log a}$. In the complex case,
$\log w$ is many-valued, so $w^z$ is many valued.

\end{document}
