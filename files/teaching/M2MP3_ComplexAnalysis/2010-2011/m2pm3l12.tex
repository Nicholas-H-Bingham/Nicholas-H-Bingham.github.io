%m2pm3l12.tex 7.2.2010 From Ingolf Becker's notes to Lectures 1-10, 2.2.2010.
\documentclass[12pt]{article}
\begin{document}
\def\ni{\noindent}
\def\i{\indent}
\def\a{\alpha}
\def\e{\epsilon}
\def\d{\delta}
\def\G{\Gamma}
\def\s{\sigma}
\def\t{\theta}
\def\z{\zeta}
\def\p{\partial}
\ni m2pm3l12.tex \\

\ni {\bf Lecture 12}.  5.2.2010.\\

\ni {\bf 2.  Complex Differentiation and the Cauchy-Riemann
Equations}\\

\ni {\it Defn}.  We say $f:{\bf C} \rightarrow {\bf C}$ is
\emph{differentiable at} $z_0$ with \emph{derivative} $w$, and write $f^{'}(z_0)=w$, if
$$
\frac{f(z)-f(z_0)}{z-z_0}\rightarrow w \hbox{ as } z\rightarrow z_0: \qquad
f^{'}(z_o)=\lim_{z\rightarrow z_0}\frac{f(z)-f(z_0)}{z-z_0}.
$$
\i The point $z_0$ is that $z$ may tend to $z_0$ in ANY way --
i.e., from ANY direction; the limit has to be the same for all
ways of approach. So,
$$
\forall\epsilon >0, \exists\delta
>0 \hbox { s.t. } \forall z \hbox{ with } |z-z_0|<\delta,
\left|\frac{f(z)-f(z_0)}{z-z_0}-f^{'}(z_0)\right|<\epsilon
$$
($arg(z-z_0)$ can be anything!). Write $z-z_0=h=k+il$ ($k$, $l$
real), $f=u+iv$ ($u$, $v$ real): $f(z)=u(x,y)+iv(x,y)$.\\
1. {h real ($l=0$)}.\\
\[\frac{u(x_0+k,y_0)-u(x_0,y_0)}{k}+i\frac{v(x_0+k,y_0)-v(x_0,y_0)}{k}\rightarrow f^{'}(z_0)
\quad (k\rightarrow 0):\]
\[u_x(x_0,y_0)+iv_x(x_0,y_0)=f^{'}(z_0),\quad\textrm{writing $u_x$ for }{\partial u}/
{\partial x}.\]
2. {h imaginary ($k=0$)}.\\
\[\frac{u(x_0,y_0+l)-u(x_0,y_0)}{l}+i\frac{v(x_0,y_0+l)-v(x_0,y_0)}{il}\rightarrow f^{'}
(z_0)\quad (l\rightarrow 0):\]
\[-iu_y(x_0,y_0)+v_y(x_0,y_0)=f^{'}(z_0),\quad\textrm{writing $u_y$ for }
{\partial u}/{\partial y}.\]
Combining, at $(x_0,y_0)$
\[u_x=v_y,\quad v_x=-u_y\]
These are called the Cauchy-Riemann Equations, C-R.\\
So differentiability at $(x_0,y_0)\Rightarrow$ C-R at $(x_0,y_0)$:
C-R are \emph{necessary} for differentiability. They are \emph{not
sufficient}. \\
{\it Example}. $f(z)=\sqrt{|xy|}\;(z=x+iy=re^{i\theta})$. So
$f,u,v\equiv 0$ on both axes. So $u_x,u_y,v_x,v_y\equiv 0$ on both
axes and C-R holds at $(0,0)$. But:
\[\frac{f(z)-f(0)}{z-0}=\frac{\sqrt{|r\cos\theta\cdot r\sin\theta|}}{re^{i\theta}}
=e^{-i\theta}\sqrt{|\cos\theta\sin\theta|}\]
RHS depends on $\theta$, i.e. \emph{how} $z=re^{i\theta} \to 0$: $f$ is \emph{not}
differentiable at $0$.\\
But there is a \emph{partial} converse:\\

\ni {\bf Theorem}.  If $f=u+iv$ and the partial derivatives
$u_x,u_y,v_x,v_y$ exist \emph{and are continuous} in a
neighbourhood of $z_0$, and satisfy the C-R equations at $z_0$,
then $f$ is dfferentiable at $z_0$.\\

\ni {\it Proof}. Take $h=k+i\ell$ so small that $z=z_0+k$ is in
the neighbourhood where partials are continuous; then
\[u(x_0+k,y_0+\ell)-u(x_0,y_0)=[u(x_0+k,y_0+\ell)-u(x_0,y_0+\ell)]
+[u(x_0,y_0+\ell)-u(x_0,y_0)]. \]
By the Mean Value Theorem (MVT):
\[\begin{array}{rcl}
[u(x_0+k,y_0+\ell)-u(x_0,y_0+\ell)]/k &=&u_x(x_0+\theta k,y_0+\ell)\\
&=&u_x(x_0,y_0)+o(1)\quad \textrm{as }h\rightarrow 0.
\end{array}\]
(here we use the $o$-notation for the error term:`$o(1)$ as
$h\rightarrow 0$' means `$\rightarrow 0$ as $h\rightarrow 0$'), by
continuity of the partial $u_x$. Similarly,
\[\begin{array}{rcll}
[u(x_0,y_0+\ell)-u(x_0,y_0)]/\ell &=&u_y(x_0,y_0+\theta^{'}\ell)&\textrm{for some }
\theta^{'}\in(0,1)\\
&=&u_y(x_0,y_0)+o(1)& \textrm{as }h\rightarrow 0.
\end{array}\]
Combining:
\[u(x_0+k,y_0+\ell)-u(x_0,y_0)=ku_x(x_0,y_0)+\ell u_y(x_0,y_0)+o(h),\]
where `$o(h)$' means `smaller order of magnitude then $h$ as
$h\rightarrow0$.' This combines two error terms, $o(k)$ and
$o(l)$, both $o(h)$ as $h^2=k^2+l^2$, $|k|\leq|h|$, $|l|\leq|h|$.
Similarly,
\[v(x_0+k,y_0+\ell)-v(x_0,y_0)=kv_x(x_0,y_0)+\ell v_y(x_0,y_0)+o(h)\].
So
\[\begin{array}{rcll}
f(z_0+h)-f(z_0)
&=&[u(x_0+k,y_0+\ell)-u(x_0,y_0)]+i[v(x_0+k,y_0+\ell)-v(x_0,y_0)]\\
&=&ku_x+{\ell u_y}+ikv_x+{i\ell v_y}+o(h).
\end{array}\]
Replace $u_y$, $v_y$ on RHS by $-v_x$, $u_x$, using the C-R equations.  The RHS becomes
$$
(k+i\ell)u_x+i(k+i\ell)v_x+o(h)\quad\textrm{by C-R} = h(u_x+iv_x)+o(h).
$$
Divide by h:
\[\begin{array}{rcll}
{f(z_0+h)-f(z_0)}/h&=&u_x(x_0,y_0)+iv_x(x_0,y_0)+o(1)\\
&\rightarrow&u_x(x_0,y_0)+iv_x(x_0,y_0)&\textrm{ as }h\rightarrow
0.
\end{array}\]
So $f^{'}(z_0)$ exists and $=u_x(x_0,y_0)+iv_x(x_0,y_0)$. // \\

\ni {\it Note}. There are \emph{three} other ways to write the RHS in the equation above.

\end{document}
